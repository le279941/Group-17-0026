% Options for packages loaded elsewhere
\PassOptionsToPackage{unicode}{hyperref}
\PassOptionsToPackage{hyphens}{url}
%
\documentclass[
]{article}
\usepackage{amsmath,amssymb}
\usepackage{iftex}
\ifPDFTeX
  \usepackage[T1]{fontenc}
  \usepackage[utf8]{inputenc}
  \usepackage{textcomp} % provide euro and other symbols
\else % if luatex or xetex
  \usepackage{unicode-math} % this also loads fontspec
  \defaultfontfeatures{Scale=MatchLowercase}
  \defaultfontfeatures[\rmfamily]{Ligatures=TeX,Scale=1}
\fi
\usepackage{lmodern}
\ifPDFTeX\else
  % xetex/luatex font selection
\fi
% Use upquote if available, for straight quotes in verbatim environments
\IfFileExists{upquote.sty}{\usepackage{upquote}}{}
\IfFileExists{microtype.sty}{% use microtype if available
  \usepackage[]{microtype}
  \UseMicrotypeSet[protrusion]{basicmath} % disable protrusion for tt fonts
}{}
\makeatletter
\@ifundefined{KOMAClassName}{% if non-KOMA class
  \IfFileExists{parskip.sty}{%
    \usepackage{parskip}
  }{% else
    \setlength{\parindent}{0pt}
    \setlength{\parskip}{6pt plus 2pt minus 1pt}}
}{% if KOMA class
  \KOMAoptions{parskip=half}}
\makeatother
\usepackage{xcolor}
\usepackage[margin=1in]{geometry}
\usepackage{color}
\usepackage{fancyvrb}
\newcommand{\VerbBar}{|}
\newcommand{\VERB}{\Verb[commandchars=\\\{\}]}
\DefineVerbatimEnvironment{Highlighting}{Verbatim}{commandchars=\\\{\}}
% Add ',fontsize=\small' for more characters per line
\usepackage{framed}
\definecolor{shadecolor}{RGB}{248,248,248}
\newenvironment{Shaded}{\begin{snugshade}}{\end{snugshade}}
\newcommand{\AlertTok}[1]{\textcolor[rgb]{0.94,0.16,0.16}{#1}}
\newcommand{\AnnotationTok}[1]{\textcolor[rgb]{0.56,0.35,0.01}{\textbf{\textit{#1}}}}
\newcommand{\AttributeTok}[1]{\textcolor[rgb]{0.13,0.29,0.53}{#1}}
\newcommand{\BaseNTok}[1]{\textcolor[rgb]{0.00,0.00,0.81}{#1}}
\newcommand{\BuiltInTok}[1]{#1}
\newcommand{\CharTok}[1]{\textcolor[rgb]{0.31,0.60,0.02}{#1}}
\newcommand{\CommentTok}[1]{\textcolor[rgb]{0.56,0.35,0.01}{\textit{#1}}}
\newcommand{\CommentVarTok}[1]{\textcolor[rgb]{0.56,0.35,0.01}{\textbf{\textit{#1}}}}
\newcommand{\ConstantTok}[1]{\textcolor[rgb]{0.56,0.35,0.01}{#1}}
\newcommand{\ControlFlowTok}[1]{\textcolor[rgb]{0.13,0.29,0.53}{\textbf{#1}}}
\newcommand{\DataTypeTok}[1]{\textcolor[rgb]{0.13,0.29,0.53}{#1}}
\newcommand{\DecValTok}[1]{\textcolor[rgb]{0.00,0.00,0.81}{#1}}
\newcommand{\DocumentationTok}[1]{\textcolor[rgb]{0.56,0.35,0.01}{\textbf{\textit{#1}}}}
\newcommand{\ErrorTok}[1]{\textcolor[rgb]{0.64,0.00,0.00}{\textbf{#1}}}
\newcommand{\ExtensionTok}[1]{#1}
\newcommand{\FloatTok}[1]{\textcolor[rgb]{0.00,0.00,0.81}{#1}}
\newcommand{\FunctionTok}[1]{\textcolor[rgb]{0.13,0.29,0.53}{\textbf{#1}}}
\newcommand{\ImportTok}[1]{#1}
\newcommand{\InformationTok}[1]{\textcolor[rgb]{0.56,0.35,0.01}{\textbf{\textit{#1}}}}
\newcommand{\KeywordTok}[1]{\textcolor[rgb]{0.13,0.29,0.53}{\textbf{#1}}}
\newcommand{\NormalTok}[1]{#1}
\newcommand{\OperatorTok}[1]{\textcolor[rgb]{0.81,0.36,0.00}{\textbf{#1}}}
\newcommand{\OtherTok}[1]{\textcolor[rgb]{0.56,0.35,0.01}{#1}}
\newcommand{\PreprocessorTok}[1]{\textcolor[rgb]{0.56,0.35,0.01}{\textit{#1}}}
\newcommand{\RegionMarkerTok}[1]{#1}
\newcommand{\SpecialCharTok}[1]{\textcolor[rgb]{0.81,0.36,0.00}{\textbf{#1}}}
\newcommand{\SpecialStringTok}[1]{\textcolor[rgb]{0.31,0.60,0.02}{#1}}
\newcommand{\StringTok}[1]{\textcolor[rgb]{0.31,0.60,0.02}{#1}}
\newcommand{\VariableTok}[1]{\textcolor[rgb]{0.00,0.00,0.00}{#1}}
\newcommand{\VerbatimStringTok}[1]{\textcolor[rgb]{0.31,0.60,0.02}{#1}}
\newcommand{\WarningTok}[1]{\textcolor[rgb]{0.56,0.35,0.01}{\textbf{\textit{#1}}}}
\usepackage{longtable,booktabs,array}
\usepackage{calc} % for calculating minipage widths
% Correct order of tables after \paragraph or \subparagraph
\usepackage{etoolbox}
\makeatletter
\patchcmd\longtable{\par}{\if@noskipsec\mbox{}\fi\par}{}{}
\makeatother
% Allow footnotes in longtable head/foot
\IfFileExists{footnotehyper.sty}{\usepackage{footnotehyper}}{\usepackage{footnote}}
\makesavenoteenv{longtable}
\usepackage{graphicx}
\makeatletter
\def\maxwidth{\ifdim\Gin@nat@width>\linewidth\linewidth\else\Gin@nat@width\fi}
\def\maxheight{\ifdim\Gin@nat@height>\textheight\textheight\else\Gin@nat@height\fi}
\makeatother
% Scale images if necessary, so that they will not overflow the page
% margins by default, and it is still possible to overwrite the defaults
% using explicit options in \includegraphics[width, height, ...]{}
\setkeys{Gin}{width=\maxwidth,height=\maxheight,keepaspectratio}
% Set default figure placement to htbp
\makeatletter
\def\fps@figure{htbp}
\makeatother
\setlength{\emergencystretch}{3em} % prevent overfull lines
\providecommand{\tightlist}{%
  \setlength{\itemsep}{0pt}\setlength{\parskip}{0pt}}
\setcounter{secnumdepth}{-\maxdimen} % remove section numbering
\ifLuaTeX
  \usepackage{selnolig}  % disable illegal ligatures
\fi
\IfFileExists{bookmark.sty}{\usepackage{bookmark}}{\usepackage{hyperref}}
\IfFileExists{xurl.sty}{\usepackage{xurl}}{} % add URL line breaks if available
\urlstyle{same}
\hypersetup{
  pdftitle={Coding Assignment 1},
  pdfauthor={Team 16},
  hidelinks,
  pdfcreator={LaTeX via pandoc}}

\title{Coding Assignment 1}
\author{Team 16}
\date{Due: 2023-10-01 23:59}

\begin{document}
\maketitle

{
\setcounter{tocdepth}{2}
\tableofcontents
}
\begin{Shaded}
\begin{Highlighting}[]
\CommentTok{\# Put any packages you want here}
\end{Highlighting}
\end{Shaded}

A Florida health insurance company wants to predict annual claims for
individual clients. The company pulls a random sample of 100 customers.
The owner wishes to charge an actuarially fair premium to ensure a
normal rate of return. The owner collects all of their current
customer's health care expenses from the last year and compares them
with what is known about each customer's plan.

The data on the 100 customers in the sample is as follows:

\begin{itemize}
\tightlist
\item
  Charges: Total medical expenses for a particular insurance plan (in
  dollars)
\item
  Age: Age of the primary beneficiary
\item
  BMI: Primary beneficiary's body mass index (kg/m2)
\item
  Female: Primary beneficiary's birth sex (0 = Male, 1 = Female)
\item
  Children: Number of children covered by health insurance plan
  (includes other dependents as well)
\item
  Smoker: Indicator if primary beneficiary is a smoker (0 = non-smoker,
  1 = smoker)
\item
  Cities: Dummy variables for each city with the default being Sanford
\end{itemize}

Answer the following questions using complete sentences and attach all
output, plots, etc. within this report.

\textbf{For this assignment, ignore the categorical variables (gender,
smoker, cities)}

\hypertarget{question-1}{%
\section{Question 1}\label{question-1}}

Perform univariate analyses on the quantitative variables (center,
shape, spread). Include descriptive statistics, and histograms. Be sure
to use terms discussed in class such as bimodal, skewed left, etc.

\begin{Shaded}
\begin{Highlighting}[]
\FunctionTok{summary}\NormalTok{(Insurance\_Data\_Group16)}
\end{Highlighting}
\end{Shaded}

\begin{verbatim}
##     Charges           Age            BMI           Children   
##  Min.   : 1256   Min.   :18.0   Min.   :17.67   Min.   :0.00  
##  1st Qu.: 5842   1st Qu.:30.5   1st Qu.:26.32   1st Qu.:0.00  
##  Median : 8825   Median :42.0   Median :30.08   Median :1.00  
##  Mean   :13819   Mean   :41.6   Mean   :30.73   Mean   :1.03  
##  3rd Qu.:18278   3rd Qu.:54.0   3rd Qu.:33.78   3rd Qu.:2.00  
##  Max.   :47928   Max.   :64.0   Max.   :47.60   Max.   :5.00
\end{verbatim}

\begin{Shaded}
\begin{Highlighting}[]
\CommentTok{\#Mean, Median, Range, SD, IQR}
\NormalTok{Insurance\_Data\_Group16 }\SpecialCharTok{\%\textgreater{}\%}
\FunctionTok{tbl\_summary}\NormalTok{(}\AttributeTok{statistic=}\FunctionTok{list}\NormalTok{(}\FunctionTok{all\_continuous}\NormalTok{() }\SpecialCharTok{\textasciitilde{}} \FunctionTok{c}\NormalTok{(}\StringTok{"\{mean\} (\{sd\})"}\NormalTok{,}
\StringTok{"\{median\} (\{p25\}, \{p75\})"}\NormalTok{,}
\StringTok{"\{min\}, \{max\}"}\NormalTok{),}
\FunctionTok{all\_categorical}\NormalTok{() }\SpecialCharTok{\textasciitilde{}} \StringTok{"\{n\} / \{N\} (\{p\}\%)"}\NormalTok{),}
\AttributeTok{type=}\FunctionTok{all\_continuous}\NormalTok{() }\SpecialCharTok{\textasciitilde{}} \StringTok{"continuous2"}
\NormalTok{)}
\end{Highlighting}
\end{Shaded}

\begin{verbatim}
## Table printed with `knitr::kable()`, not {gt}. Learn why at
## https://www.danieldsjoberg.com/gtsummary/articles/rmarkdown.html
## To suppress this message, include `message = FALSE` in code chunk header.
\end{verbatim}

\begin{longtable}[]{@{}lc@{}}
\toprule\noalign{}
\textbf{Characteristic} & \textbf{N = 100} \\
\midrule\noalign{}
\endhead
\bottomrule\noalign{}
\endlastfoot
Charges & \\
Mean (SD) & 13,819 (12,359) \\
Median (IQR) & 8,825 (5,842, 18,278) \\
Range & 1,256, 47,928 \\
Age & \\
Mean (SD) & 42 (14) \\
Median (IQR) & 42 (31, 54) \\
Range & 18, 64 \\
BMI & \\
Mean (SD) & 31 (6) \\
Median (IQR) & 30 (26, 34) \\
Range & 18, 48 \\
Children & \\
0 & 48 / 100 (48\%) \\
1 & 18 / 100 (18\%) \\
2 & 21 / 100 (21\%) \\
3 & 11 / 100 (11\%) \\
5 & 2 / 100 (2.0\%) \\
\end{longtable}

\begin{Shaded}
\begin{Highlighting}[]
\CommentTok{\#Histograms}
\NormalTok{Charges}\OtherTok{\textless{}{-}}\FunctionTok{rnorm}\NormalTok{(}\DecValTok{100}\NormalTok{,}\AttributeTok{mean=}\DecValTok{13819}\NormalTok{,}\AttributeTok{sd=}\DecValTok{12359}\NormalTok{)}
\FunctionTok{hist}\NormalTok{(Charges, }\AttributeTok{col=}\StringTok{\textquotesingle{}hotpink\textquotesingle{}}\NormalTok{,}\AttributeTok{border=}\StringTok{\textquotesingle{}white\textquotesingle{}}\NormalTok{)}
\end{Highlighting}
\end{Shaded}

\includegraphics{CodingAssignment01_files/figure-latex/q1-1.pdf}

\begin{Shaded}
\begin{Highlighting}[]
\NormalTok{Age}\OtherTok{\textless{}{-}}\FunctionTok{rnorm}\NormalTok{(}\DecValTok{100}\NormalTok{,}\AttributeTok{mean=}\DecValTok{42}\NormalTok{,}\AttributeTok{sd=}\DecValTok{14}\NormalTok{)}
\FunctionTok{hist}\NormalTok{(Age, }\AttributeTok{col=}\StringTok{\textquotesingle{}hotpink\textquotesingle{}}\NormalTok{,}\AttributeTok{border=}\StringTok{\textquotesingle{}white\textquotesingle{}}\NormalTok{)}
\end{Highlighting}
\end{Shaded}

\includegraphics{CodingAssignment01_files/figure-latex/q1-2.pdf}

\begin{Shaded}
\begin{Highlighting}[]
\NormalTok{BMI}\OtherTok{\textless{}{-}}\FunctionTok{rnorm}\NormalTok{(}\DecValTok{100}\NormalTok{,}\AttributeTok{mean=}\DecValTok{31}\NormalTok{,}\AttributeTok{sd=}\DecValTok{6}\NormalTok{)}
\FunctionTok{hist}\NormalTok{(BMI, }\AttributeTok{col=}\StringTok{\textquotesingle{}hotpink\textquotesingle{}}\NormalTok{,}\AttributeTok{border=}\StringTok{\textquotesingle{}white\textquotesingle{}}\NormalTok{)}
\end{Highlighting}
\end{Shaded}

\includegraphics{CodingAssignment01_files/figure-latex/q1-3.pdf}

\begin{Shaded}
\begin{Highlighting}[]
\NormalTok{Children}\OtherTok{\textless{}{-}}\FunctionTok{rnorm}\NormalTok{(}\DecValTok{100}\NormalTok{,}\AttributeTok{mean=}\NormalTok{,}\AttributeTok{sd=}\NormalTok{)}
\FunctionTok{hist}\NormalTok{(Children, }\AttributeTok{col=}\StringTok{\textquotesingle{}hotpink\textquotesingle{}}\NormalTok{,}\AttributeTok{border=}\StringTok{\textquotesingle{}white\textquotesingle{}}\NormalTok{)}
\end{Highlighting}
\end{Shaded}

\includegraphics{CodingAssignment01_files/figure-latex/q1-4.pdf}

\begin{Shaded}
\begin{Highlighting}[]
\CommentTok{\#Boxplots}
\FunctionTok{boxplot}\NormalTok{(Insurance\_Data\_Group16}\SpecialCharTok{$}\NormalTok{Charges)}
\end{Highlighting}
\end{Shaded}

\includegraphics{CodingAssignment01_files/figure-latex/q1-5.pdf}

\begin{Shaded}
\begin{Highlighting}[]
\FunctionTok{boxplot}\NormalTok{(Insurance\_Data\_Group16}\SpecialCharTok{$}\NormalTok{Age)}
\end{Highlighting}
\end{Shaded}

\includegraphics{CodingAssignment01_files/figure-latex/q1-6.pdf}

\begin{Shaded}
\begin{Highlighting}[]
\FunctionTok{boxplot}\NormalTok{(Insurance\_Data\_Group16}\SpecialCharTok{$}\NormalTok{BMI)}
\end{Highlighting}
\end{Shaded}

\includegraphics{CodingAssignment01_files/figure-latex/q1-7.pdf}

\begin{Shaded}
\begin{Highlighting}[]
\FunctionTok{boxplot}\NormalTok{(Insurance\_Data\_Group16}\SpecialCharTok{$}\NormalTok{Children)}
\end{Highlighting}
\end{Shaded}

\includegraphics{CodingAssignment01_files/figure-latex/q1-8.pdf}

\hypertarget{question-2}{%
\section{Question 2}\label{question-2}}

Perform bivariate analyses on the quantitative variables (direction,
strength and form). Describe the linear association between all
variables.

\begin{Shaded}
\begin{Highlighting}[]
\CommentTok{\#Distribution and relationships of each variable}
\FunctionTok{pairs}\NormalTok{(Insurance\_Data\_Group16, }\AttributeTok{col=}\StringTok{"hotpink"}\NormalTok{)}
\end{Highlighting}
\end{Shaded}

\includegraphics{CodingAssignment01_files/figure-latex/q2-1.pdf}

\begin{Shaded}
\begin{Highlighting}[]
\CommentTok{\#Correlation}
\CommentTok{\#Correlation between Charges and Children}
\FunctionTok{cor}\NormalTok{(Insurance\_Data\_Group16}\SpecialCharTok{$}\NormalTok{Charges, Insurance\_Data\_Group16}\SpecialCharTok{$}\NormalTok{Children)}
\end{Highlighting}
\end{Shaded}

\begin{verbatim}
## [1] 0.08724211
\end{verbatim}

\begin{Shaded}
\begin{Highlighting}[]
\FunctionTok{corrplot}\NormalTok{(}\FunctionTok{cor}\NormalTok{(Insurance\_Data\_Group16),}
\AttributeTok{type=}\StringTok{"lower"}\NormalTok{,}
\AttributeTok{order=}\StringTok{"hclust"}\NormalTok{,}
\AttributeTok{tl.col=}\StringTok{"black"}\NormalTok{,}
\AttributeTok{tl.srt=}\DecValTok{45}\NormalTok{,}
\AttributeTok{addCoef.col =} \StringTok{"black"}\NormalTok{,}
\AttributeTok{diag=}\ConstantTok{FALSE}\NormalTok{)}

\CommentTok{\#Correlation between Age and BMI}
\FunctionTok{cor}\NormalTok{(Insurance\_Data\_Group16}\SpecialCharTok{$}\NormalTok{Age, Insurance\_Data\_Group16}\SpecialCharTok{$}\NormalTok{BMI)}
\end{Highlighting}
\end{Shaded}

\begin{verbatim}
## [1] 0.09387806
\end{verbatim}

\begin{Shaded}
\begin{Highlighting}[]
\FunctionTok{corrplot}\NormalTok{(}\FunctionTok{cor}\NormalTok{(Insurance\_Data\_Group16),}
\AttributeTok{type=}\StringTok{"lower"}\NormalTok{,}
\AttributeTok{order=}\StringTok{"hclust"}\NormalTok{,}
\AttributeTok{tl.col=}\StringTok{"black"}\NormalTok{,}
\AttributeTok{tl.srt=}\DecValTok{45}\NormalTok{,}
\AttributeTok{addCoef.col =} \StringTok{"black"}\NormalTok{,}
\AttributeTok{diag=}\ConstantTok{FALSE}\NormalTok{)}
\end{Highlighting}
\end{Shaded}

\includegraphics{CodingAssignment01_files/figure-latex/q2-2.pdf}

\begin{Shaded}
\begin{Highlighting}[]
\CommentTok{\#Regression }
\CommentTok{\#fit simple linear regression model}
\NormalTok{fit}\OtherTok{\textless{}{-}}\FunctionTok{lm}\NormalTok{(Charges}\SpecialCharTok{\textasciitilde{}}\NormalTok{Children, }\AttributeTok{data=}\NormalTok{Insurance\_Data\_Group16)}
\FunctionTok{summary}\NormalTok{(fit)}
\end{Highlighting}
\end{Shaded}

\begin{verbatim}
## 
## Call:
## lm(formula = Charges ~ Children, data = Insurance_Data_Group16)
## 
## Residuals:
##    Min     1Q Median     3Q    Max 
## -11960  -8647  -4480   4501  34136 
## 
## Coefficients:
##             Estimate Std. Error t value Pr(>|t|)    
## (Intercept)  12901.2     1628.6   7.921 3.73e-12 ***
## Children       891.2     1028.0   0.867    0.388    
## ---
## Signif. codes:  0 '***' 0.001 '**' 0.01 '*' 0.05 '.' 0.1 ' ' 1
## 
## Residual standard error: 12370 on 98 degrees of freedom
## Multiple R-squared:  0.007611,   Adjusted R-squared:  -0.002515 
## F-statistic: 0.7516 on 1 and 98 DF,  p-value: 0.3881
\end{verbatim}

\begin{Shaded}
\begin{Highlighting}[]
\CommentTok{\#fit simple linear regression model}
\NormalTok{fit}\OtherTok{\textless{}{-}}\FunctionTok{lm}\NormalTok{(Age}\SpecialCharTok{\textasciitilde{}}\NormalTok{BMI, }\AttributeTok{data=}\NormalTok{Insurance\_Data\_Group16)}
\FunctionTok{summary}\NormalTok{(fit)}
\end{Highlighting}
\end{Shaded}

\begin{verbatim}
## 
## Call:
## lm(formula = Age ~ BMI, data = Insurance_Data_Group16)
## 
## Residuals:
##      Min       1Q   Median       3Q      Max 
## -25.2644 -10.3090   0.9644  12.4725  23.5304 
## 
## Coefficients:
##             Estimate Std. Error t value Pr(>|t|)    
## (Intercept)  34.8271     7.3934   4.711 8.15e-06 ***
## BMI           0.2204     0.2361   0.933    0.353    
## ---
## Signif. codes:  0 '***' 0.001 '**' 0.01 '*' 0.05 '.' 0.1 ' ' 1
## 
## Residual standard error: 14.2 on 98 degrees of freedom
## Multiple R-squared:  0.008813,   Adjusted R-squared:  -0.001301 
## F-statistic: 0.8714 on 1 and 98 DF,  p-value: 0.3529
\end{verbatim}

\begin{Shaded}
\begin{Highlighting}[]
\CommentTok{\#Scatterplots}
\CommentTok{\#Scatterplot(s) of Charges vs. Number of Children}
\FunctionTok{ggplot}\NormalTok{(Insurance\_Data\_Group16,}\FunctionTok{aes}\NormalTok{(Charges,Children))}\SpecialCharTok{+}
  \FunctionTok{geom\_point}\NormalTok{(}\AttributeTok{color=}\StringTok{\textquotesingle{}hotpink\textquotesingle{}}\NormalTok{)}\SpecialCharTok{+}
  \FunctionTok{theme\_bw}\NormalTok{()}\SpecialCharTok{+}
  \FunctionTok{labs}\NormalTok{(}\AttributeTok{title=}\StringTok{"Medical Expenses \& Children of the Beneficiary"}\NormalTok{)}
\end{Highlighting}
\end{Shaded}

\includegraphics{CodingAssignment01_files/figure-latex/q2-3.pdf}

\begin{Shaded}
\begin{Highlighting}[]
\CommentTok{\#Scatterplot of Age vs. BMI}
\FunctionTok{ggplot}\NormalTok{(Insurance\_Data\_Group16,}\FunctionTok{aes}\NormalTok{(Age,BMI))}\SpecialCharTok{+}
  \FunctionTok{geom\_point}\NormalTok{(}\AttributeTok{color=}\StringTok{\textquotesingle{}hotpink\textquotesingle{}}\NormalTok{)}\SpecialCharTok{+}
  \FunctionTok{theme\_bw}\NormalTok{()}\SpecialCharTok{+}
  \FunctionTok{labs}\NormalTok{(}\AttributeTok{title=}\StringTok{"Data by Age \& BMI"}\NormalTok{)}
\end{Highlighting}
\end{Shaded}

\includegraphics{CodingAssignment01_files/figure-latex/q2-4.pdf}

\hypertarget{question-3}{%
\section{Question 3}\label{question-3}}

Generate a regression equation in the following form:

\[Charges = \beta_{0}+\beta_{1}*Age+\beta_{2}*BMI+\beta_{3}*Children\]

\begin{Shaded}
\begin{Highlighting}[]
\NormalTok{model}\OtherTok{\textless{}{-}}\FunctionTok{lm}\NormalTok{(Charges}\SpecialCharTok{\textasciitilde{}}\NormalTok{Age}\SpecialCharTok{+}\NormalTok{BMI}\SpecialCharTok{+}\NormalTok{Children, }\AttributeTok{data=}\NormalTok{Insurance\_Data\_Group16)}
\FunctionTok{summary}\NormalTok{(model)}
\end{Highlighting}
\end{Shaded}

\begin{verbatim}
## 
## Call:
## lm(formula = Charges ~ Age + BMI + Children, data = Insurance_Data_Group16)
## 
## Residuals:
##    Min     1Q Median     3Q    Max 
## -13676  -7061  -4564   7852  29571 
## 
## Coefficients:
##             Estimate Std. Error t value Pr(>|t|)    
## (Intercept) -13486.5     6502.4  -2.074   0.0407 *  
## Age            360.6       78.9   4.570 1.45e-05 ***
## BMI            359.1      184.8   1.943   0.0549 .  
## Children      1233.0      921.8   1.338   0.1842    
## ---
## Signif. codes:  0 '***' 0.001 '**' 0.01 '*' 0.05 '.' 0.1 ' ' 1
## 
## Residual standard error: 11070 on 96 degrees of freedom
## Multiple R-squared:  0.2225, Adjusted R-squared:  0.1982 
## F-statistic: 9.158 on 3 and 96 DF,  p-value: 2.178e-05
\end{verbatim}

also write out the regression cleanly in this document.

\hypertarget{question-4}{%
\section{Question 4}\label{question-4}}

An eager insurance representative comes back with a potential client.
The client is 40, their BMI is 30, and they have one dependent. Using
the regression equation above, predict the amount of medical expenses
associated with this policy. (Provide a 95\% confidence interval as
well)

\begin{Shaded}
\begin{Highlighting}[]
\NormalTok{newPrediction}\OtherTok{\textless{}{-}}\FunctionTok{data.frame}\NormalTok{(}\AttributeTok{Age=}\DecValTok{40}\NormalTok{,}
                          \AttributeTok{BMI=}\DecValTok{30}\NormalTok{,}
                          \AttributeTok{Children=}\DecValTok{1}\NormalTok{)}
\FunctionTok{predict}\NormalTok{(model, }\AttributeTok{newdata=}\NormalTok{newPrediction)}
\end{Highlighting}
\end{Shaded}

\begin{verbatim}
##        1 
## 12943.56
\end{verbatim}

\end{document}
